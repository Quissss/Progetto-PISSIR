\documentclass{article}
\usepackage{graphicx}
\usepackage[italian]{babel}
\usepackage{float}
\usepackage{hyperref}
\usepackage{listings}
\usepackage{color}
\usepackage{longtable}

\definecolor{bluekeywords}{rgb}{0.13,0.13,1}
\definecolor{greencomments}{rgb}{0,0.5,0}
\definecolor{redstrings}{rgb}{0.9,0,0}

\usepackage{listings}
\lstset{language=[Sharp]C,
showspaces=false,
showtabs=false,
breaklines=true,
showstringspaces=false,
breakatwhitespace=true,
escapeinside={(*@}{@*)},
commentstyle=\color{greencomments},
keywordstyle=\color{bluekeywords}\bfseries,
stringstyle=\color{redstrings},
basicstyle=\ttfamily
}
\hypersetup{
    colorlinks=true,
    linkcolor=blue,
    filecolor=magenta,      
    urlcolor=cyan,
    pdfpagemode=FullScreen,
    }

\title{Relazione per il progetto di Progettazione e Implementazione dei Sistemi Software in Rete}
\author{
  Luca Benetti\\
  20043903
  \and
  Anton Borislavov Iliev\\
  20035170
  \and
  Linda Monfermoso\\
  20028464
}

\begin{document}

\maketitle

\section{Introduzione}
Come previsto dal progetto presentato durante il corso, abbiamo creato un applicativo web pensato per gestire le ricariche di automobili elettriche, in più parcheggi, tramite robot autonomi.

Il progetto è stato realizzato in \href{https://dotnet.microsoft.com/en-us/apps/aspnet}{ASP.NET Core} (linguaggio C\#), su \href{https://dotnet.microsoft.com/en-us/download/dotnet/8.0}{.NET Core 8} e \href{https://learn.microsoft.com/en-us/aspnet/entity-framework}{Entity Framework}. Per l'interfaccia grafica è stato utilizzato \href{https://learn.microsoft.com/en-us/aspnet/core/razor-pages/?view=aspnetcore-8.0&tabs=visual-studio}{Razor}, mentre per il database \href{https://www.sqlite.org/index.html}{Sqlite} (con Entity Framework Core) e \href{https://learn.microsoft.com/en-us/dotnet/csharp/linq/}{LinQ}. La comunicazione con il database da parte delle varie funzioni e servizi avviene tramite \href{https://learn.microsoft.com/it-it/dotnet/architecture/microservices/microservice-ddd-cqrs-patterns/infrastructure-persistence-layer-design}{repository pattern}.

I programmi CamSimulator e MonitorSimulator sono stati realizzati tramite \href{https://learn.microsoft.com/en-us/visualstudio/ide/create-csharp-winform-visual-studio?view=vs-2022}{Windows Form}, e le funzionalità dei MwBot sono state implementate tramite protocollo MQTT, grazie alla libreria \href{https://github.com/dotnet/MQTTnet}{MQTTnet}.

Per comunicare con gli utenti è stata utilizzata l'API di \href{https://github.com/TelegramBots/Telegram.Bot}{Telegram}.
\section{Specifica}
\subsection{Casi d'uso e requisiti}
Il diagramma dei casi d'uso è disponibile nel \href{run:./casi d'uso.pdf}{file PDF apposito}.
\subsubsection{Descrizione casi d'uso e requisiti funzionali}

    \begin{longtable}{|c|p{3cm}|p{7cm}|}
        \hline
         1 & Login & L'utente si autentica alla piattaforma \\
         \hline
         2 & Creazione utente & Viene creato un utente nella piattaforma \\
         \hline
         3 & Aggiorna dati utente & L'utente aggiorna i propri dati \\
         \hline
         4 & Effettua pagamento & L'utente effettua un pagamento in seguito a una ricarica \\
         \hline
         5 & Visualizza pagamento & L'utente visualizza il pagamento da effettuare \\
         \hline
         6 & Creazione pagamento & Viene creato un pagamento in seguito a una ricarica finita \\
         \hline
         7 & Crea prenotazione & L'utente premium crea una prenotazione \\
         \hline
         8 & Elimina prenotazione & L'utente premium elimina una prenotazione \\
         \hline
         9 & Modifica prenotazione & L'utente premium modifica una prenotazione \\
         \hline
         10 & Elenco prenotazioni & L'amministratore visualizza elenco delle prenotazioni \\
         \hline
         11 & Invio messaggio utente & Il sistema invia un messaggio (via Telegram) all'utente \\
         \hline
         12 & Invio messaggio robot & Il sistema invia un messaggio (via MQTT) all'utente \\
         \hline
         13 & Aggiunta/rimozione robot & L'amministratore aggiunge o rimuove un MWbot \\
         \hline
         14 & Monitoraggio occupazione & Il sistema monitora le macchine in entrata e uscita per determinare occupazione dei posti \\
         \hline
         15 & Aggiunta/rimozione auto & L'utente rimuove o aggiunge un'automobile \\
         \hline
         16 & Aggiungi/rimuovi macchina coda ricariche & Il sistema aggiorna lo stato delle macchine in coda per ricarica \\
         \hline
         17 & Elenco coda ricariche & L'amministratore visualizza l'elenco delle ricariche in coda \\
         \hline
         18 & Aggiornamento costi ricarica & L'amministratore aggiorna i costi delle ricariche \\
         \hline
         20 & Elenco pagamenti & L'amministratore visualizza l'elenco dei pagamenti \\
         \hline
         21 & Rimozione utente & L'amministratore rimuove un utente dalla piattaforma \\
         \hline
         22 & Elenco posteggi & L'amministratore visualizza un elenco dei posteggi di un parcheggio \\
         \hline
         23 & Elenco auto & L'amministratore visualizza un elenco delle auto registrate alla piattaforma \\
         \hline
         24 & Ricerca utente & L'amministratore ricerca un utente registrato \\
         \hline
         25 & Ricerca parcheggio & L'amministratore ricerca un parcheggio \\
         \hline
    \end{longtable}
\subsubsection{Requisiti non funzionali}
\begin{itemize}
    \item L'interfaccia è grafica e realizzata con Razor Pages
    \item Il database è realizzato con SQLite
    \item Le specifiche di progettazione sono realizzate con diagrammi UML
    \item Il sistema è implementato in .NET e EntityFramework
    \item La password è lunga 8 caratteri, con maiusole, minuscole e numero
    \item Le date sono memorizzate nel formato standard UTC
    \item Lo scambio di messaggi con MWbot avviene tramite MQTT
    \item Lo scambio di messaggi tra sistema di prenotazione e ricariche e utente avviene tramite Telegram
    \item Il pagamento avviene tramite PayPal
    \item La registrazione e l'autenticazione sono gestite dalla libreria Identity
    \item Il rilevamento dei posti occupati è gestito da un sensore posto all'entrata del parcheggio
\end{itemize}
\subsection{Diagramma delle classi di dominio}
Il diagramma delle classi di dominio è disponibile nel \href{run:./dominio.pdf}{file PDF apposito}.
\section{Progettazione}
\subsection{Diagramma delle classi}
I diagrammi di tutte le classi rilevanti per il progetto sono presenti nella cartella \texttt{Diagrammi Classe}. 
\subsection{Documentazione API}
Per documentare le API abbiamo utilizzato \href{https://swagger.io}{Swagger}. La lista di API utilizzate nel progetto è visibile alla pagina \url{https://localhost:7237/swagger/index.html} quando questo è in esecuzione. È anche presente un documento in formato PDF (\href{run:./api-documentation.pdf}{api-documentation}) qualora non fosse possibile visualizzare la pagina online.
\subsection{MQTT}
La gestione e comunicazione con i robot MWbot è stata implementata mediante il protocollo MQTT. All'avvio, l'applicativo verifica gli MwBot che sono attualmente online e li istanzia come client, connettendoli al broker ed effettuando l'inizializzazione dei parametri.

Come da protocollo, la nostra implementazione prevede la presenza di un broker e di più client, che rappresentano gli MWbot. In particolare, il broker è responsabile della ricezione e distribuzione di messaggi tra client MWbot e ne gestisce le richieste. Il client, invece, comunica con il broker pubblicando sui topic pertinenti.

\subsubsection{Struttura messaggi MQTT}
Classe rappresentante un messaggio MQTT:
\begin{lstlisting}[language=csh]
    public class MqttClientMessage : MwBot
{
    public MessageType MessageType { get; set; }
    public int? ParkingSlotId { get; set; }
    public ParkingSlot? ParkingSlot { get; set; }
    public decimal? CurrentCarCharge { get; set; }
    public decimal? TargetBatteryPercentage { get; set; }
    public string? UserId { get; set; }
    public string? CarPlate { get; set; }
    public CurrentlyCharging? CurrentlyCharging { get; set; }
    public int? ImmediateRequestId { get; set; }
    public ImmediateRequest? ImmediateRequest { get; set; }
}
\end{lstlisting}
Tipi di messaggi MQTT da client a broker:
\begin{itemize}
    \item RequestCharge: il MwBot chiede al broker di ricaricare un'auto
    \item CompleteCharge: il MwBot riferisce al broker che l'auto è completamente carica
    \item UpdateCharging: il MwBot riferisce al broker la sua percentuale di carica
    \item UpdateMwBot: il MwBot riferisce al broker il suo stato (StandBy, MovingToSlot, MovingToDock, Offline, ChargingCar, Recharging)
    \item RequestRecharge: il MwBot richiede al broker di ricaricarsi al dock
    \item DisconnectClient: il MwBot chiede di essere disconnesso
    \item RequestMwBot: connette client a MwBot
\end{itemize}
Tipi di messaggi MQTT da broker a client:
\begin{itemize}
    \item StartCharging: indica al MwBot di iniziare la ricarica di un'auto
    \item StartRecharge: indica al MwBot di iniziare la ricarica della propria batteria al dock
    \item ChargeCompleted: indica al MwBot che la ricarica dell'auto è completa, in quanto percentuale specificata dall'utente
    \item ReturnMwBot: indica al MwBot di ritornare al dock
    \item StopCharging: indica al MwBot di interrompere il processo di ricarica
\end{itemize}
L'emulazione del movimento, della ricarica delle auto e della ricarica degli MwBot è simulato nella classe \texttt{MqttMwBotClient}, nelle funzioni \texttt{SimulateMovement}, \texttt{SimulateChargingProcess} e \texttt{SimulateRechargingProcess}.
\section{Implementazione}
\subsection{Istruzioni di installazione}
Si consiglia l'utilizzo del programma Visual Studio per avviare l'applicativo web e i due programmi che permettono di emulare l'entrata delle auto nel parcheggio (CamSimulator) e l'occupazione dei posti auto (MonitorSimulator).

Una volta aperta la soluzione con Visual Studio, è necessario aggiungere i segreti utente. Per farlo, cliccare su \texttt{Progetto.App}, selezionare \texttt{Gestione segreti utente}, e aggiungere il codice presente nel file \href{run:./segreti.json}{segreti.json}.

Aggiunti i segreti, è sufficiente fare click destro su \texttt{Solution Pissir.Progetto} nella schermata \texttt{Solution Explorer}, selezionare \texttt{Configure startp projects...}, selezionare tutti i progetti e premere su \texttt{Start} nella schermata principale.

Avviata la soluzione, è possibile utilizzare l'applicativo web recandosi all'indirizzo \url{https://localhost:7237}, emulare l'entrata/uscita delle auto tramite CamSimulator e visionare lo stato di occupazione dei posteggi tramite MonitorSimulator.

Utenze:
\begin{itemize}
    \item admin@admin.it - Admin1234\$ 
    \item utente01@utente.com - bHw2Hs3!4bzq-aQ
    \item utentepremium01@upremium.com - SgRE5GZp\_rj6s2E
\end{itemize}
\subsection{Progetto.App}
Progetto.App implementa l'applicativo web. Contiene configurazione all'avvio, log, controller con i vari endpoint con cui interfacciarsi per le richieste e le pagine front-end di interfaccia.

Sono presenti tre tipi di utente:
\begin{itemize}
    \item Utente base: può visionare i parcheggi disponibili, il loro stato d'occupazione ed effettuare ricariche e/o soste scegliendo dal menù “Dashboard $>$ Services” quando la sua auto viene rilevata dalla telecamera in entrata;
    \item Utente premium: come utente base, ma può prenotare una ricarica tramite il menù
    “Dashboard $>$ Reservation”;
    \item Utente amministratore: ha il controllo completo sui parcheggi; può quindi effettuare alcune operazioni CRUD non-critical tramite i vari menù, accendere / spegnere i robot, filtrare i pagamenti effettuati.
\end{itemize}
\subsection{Progetto.App.Core}
Progetto.App Core (domain + application) contiene la configurazione del database, i modelli utilizzati dall'applicativo web, le migration, i validator dei modelli, le repository per interfacciarsi con il database e i servizi.
\subsection{CamSimulator}
CamSimulator è un'applicazione di tipo Windows Form che rappresenta la telecamera con
rilevamento targa presente all'entrata di uno specifico parcheggio (da selezionare).
Rileva targhe in entrata / uscita: in seguito all'entrata, viene richiesto all'utente sulla pagina Dashboard/Servizi se effettuare una sosta o una ricarica. Si interfaccia con l'applicazione web tramite Post e Get a due endpoint ApiRest presenti in un
controller dedicato.
\subsection{MonitorSimulator}
Monitor Simulator è un'applicazione di tipo Windows Form che rappresenta i sensori presenti nei posti auto. Tramite essa è possibile controllare l'occupazione dei posti auto in vari parcheggi.
\subsection{PayPal.REST}
Implementa la logica dei pagamenti tramite PayPal. È una API di tipo REST.
\end{document}
