\documentclass{article}
\usepackage{graphicx}
\usepackage{babel}[italian]
\usepackage{float}
\usepackage{hyperref}
\hypersetup{
    colorlinks=true,
    linkcolor=blue,
    filecolor=magenta,      
    urlcolor=cyan,
    pdfpagemode=FullScreen,
    }

\title{Relazione per il progetto di Progettazione e Implementazione dei Sistemi Software in Rete}
\author{
  Luca Benetti\\
  20043903
  \and
  Anton Borislavov Iliev\\
  20035170
  \and
  Linda Monfermoso\\
  20028464
}

\begin{document}

\maketitle

\section{Introduzione}
Come previsto dal progetto presentato durante il corso, abbiamo creato un applicativo web pensato per gestire le ricariche di automobili elettriche, in più parcheggi, tramite robot autonomi.
\section{Specifica}
\subsection{Casi d'uso e requisiti}
Il diagramma dei casi d'uso è disponibile nel \href{run:./casi d'uso.pdf}{file PDF apposito}.
\subsubsection{Descrizione casi d'uso e requisiti funzionali}
\begin{table}[H]
    \centering
    \begin{tabular}{|c|p{3cm}|p{7cm}|}
        \hline
         1 & Login & L'utente si autentica alla piattaforma \\
         \hline
         2 & Creazione utente & Viene creato un utente nella piattaforma \\
         \hline
         3 & Aggiorna dati utente & L'utente aggiorna i propri dati \\
         \hline
         4 & Effettua pagamento & L'utente effettua un pagamento in seguito a una ricarica \\
         \hline
         5 & Visualizza pagamento & L'utente visualizza il pagamento da effettuare \\
         \hline
         6 & Creazione pagamento & Viene creato un pagamento in seguito a una ricarica finita \\
         \hline
         7 & Crea prenotazione & L'utente premium crea una prenotazione \\
         \hline
         8 & Elimina prenotazione & L'utente premium elimina una prenotazione \\
         \hline
         9 & Modifica prenotazione & L'utente premium modifica una prenotazione \\
         \hline
         10 & Elenco prenotazioni & L'amministratore visualizza elenco delle prenotazioni \\
         \hline
         11 & Invio messaggio utente & Il sistema invia un messaggio (via Telegram) all'utente \\
         \hline
         12 & Invio messaggio robot & Il sistema invia un messaggio (via MQTT) all'utente \\
         \hline
         13 & Aggiunta/rimozione robot & L'amministratore aggiunge o rimuove un MWbot \\
         \hline
         14 & Monitoraggio occupazione & Il sistema monitora le macchine in entrata e uscita per determinare occupazione dei posti \\
         \hline
         15 & Aggiunta/rimozione auto & L'utente rimuove o aggiunge un'automobile \\
         \hline
         16 & Aggiungi/rimuovi macchina coda ricariche & Il sistema aggiorna lo stato delle macchine in coda per ricarica \\
         \hline
         17 & Elenco coda ricariche & L'amministratore visualizza l'elenco delle ricariche in coda \\
         \hline
         18 & Aggiornamento costi ricarica & L'amministratore aggiorna i costi delle ricariche \\
         \hline
         20 & Elenco pagamenti & L'amministratore visualizza l'elenco dei pagamenti \\
         \hline
         21 & Rimozione utente & L'amministratore rimuove un utente dalla piattaforma \\
         \hline
         22 & Elenco posteggi & L'amministratore visualizza un elenco dei posteggi di un parcheggio \\
         \hline
         23 & Elenco auto & L'amministratore visualizza un elenco delle auto registrate alla piattaforma \\
         \hline
         24 & Ricerca utente & L'amministratore ricerca un utente registrato \\
         \hline
         25 & Ricerca parcheggio & L'amministratore ricerca un parcheggio \\
         \hline
    \end{tabular}
\end{table}
\subsubsection{Requisiti non funzionali}
\begin{itemize}
    \item L'interfaccia è grafica e realizzata con Razor Pages
    \item Il database è realizzato con SQLite
    \item Le specifiche di progettazione sono realizzate con diagrammi UML
    \item Il sistema è implementato in .NET e EntityFramework
    \item La password è lunga 8 caratteri, con maiusole, minuscole e numero
    \item Le date sono memorizzate nel formato standard UTC
    \item Lo scambio di messaggi con MWbot avviene tramite MQTT
    \item Lo scambio di messaggi tra sistema di prenotazione e ricariche e utente avviene tramite Telegram
    \item Il pagamento avviene tramite PayPal
    \item La registrazione e l'autenticazione sono gestite dalla libreria Identity
    \item Il rilevamento dei posti occupati è gestito da un sensore posto all'entrata del parcheggio
\end{itemize}
\subsection{Diagramma delle classi di dominio}
Il diagramma delle classi di dominio è disponibile nel \href{run:./dominio.pdf}{file PDF apposito}.
\section{Progettazione}
\subsection{Diagramma delle classi}

\subsection{Documentazione API}
Per documentare le API abbiamo utilizzato \href{https://swagger.io}{Swagger}. La lista di API utilizzate nel progetto è visibile alla pagina \url{https://localhost:7237/swagger/index.html} quando questo è in esecuzione.
\subsection{MQTT}
La gestione e comunicazione con i robot MWbot è stata implementata mediante il protocollo MQTT. All'avvio, l'applicativo verifica gli MwBot che sono attualmente online e li istanzia come client, connettendoli al broker ed effettuando l'inizializzazione dei parametri.

Come da protocollo, la nostra implementazione prevede la presenza di un broker e di più client, che rappresentano gli MWbot. In particolare, il broker è responsabile della ricezione e distribuzione di messaggi tra client MWbot e ne gestisce le richieste. Il client, invece, comunica con il broker pubblicando sui topic pertinenti.


\section{Implementazione}
\subsection{Istruzioni di installazione}

\subsection{Suddivisione compiti}

\end{document}
